% this is <libstfinv.tex>
% ----------------------------------------------------------------------------
% 
% Copyright (c) 2013 by Thomas Forbriger (BFO Schiltach) 
% 
% libstfinv
% 
% REVISIONS and CHANGES 
%    13/02/2013   V1.0   Thomas Forbriger
% 
% ============================================================================
%
\documentclass{article}
\newcommand{\version}{2013/02/13 V1.0}
\usepackage{ngerman}
\usepackage{pslatex}
\usepackage{anysize}
\usepackage{amsmath}
\usepackage{graphicx}
\selectlanguage{english}
%----------------------------------------------------------------------
\newcommand{\libstfinv}{\texttt{libstfinv}}
\newcommand{\Fourier}[1]{\ensuremath{\tilde{#1}}}
\newcommand{\Fd}{\ensuremath{\Fourier{d}}}
%\newcommand{\Fs}{\ensuremath{\Fourier{s}}}
%\newcommand{\Sg}{\ensuremath{g}}
%\newcommand{\Fq}{\ensuremath{\Fourier{q}}}
\newcommand{\Fs}{\ensuremath{\Fourier{g}}}
\newcommand{\Sg}{\ensuremath{s}}
\newcommand{\Fq}{\ensuremath{\Fourier{c}}}
\newcommand{\Fg}{\ensuremath{\Fourier{\Sg}}}
\newcommand{\FQl}{\ensuremath{\Fourier{\Sg}_{l}^{\text{opt}}}}
\newcommand{\So}{\ensuremath{\omega}}
\newcommand{\Sf}{\ensuremath{f}}
\newcommand{\Sr}{\ensuremath{r}}
\newcommand{\Ssk}{\ensuremath{\sum\limits_{k=1}^{M}}}
\newcommand{\SslN}{\ensuremath{\sum\limits_{l=0}^{N-1}}}
\newcommand{\Silk}{\ensuremath{_{lk}}}
\newcommand{\Scc}{\ensuremath{^{\ast}}}
\newcommand{\SmE}{\ensuremath{\overline{E}}}
\newcommand{\Se}{\ensuremath{\epsilon}}
%----------------------------------------------------------------------
\begin{document}
\title{\libstfinv}
\author{Thomas Forbriger\\ \texttt{\version}}
\date{(\today, BFO Schiltach)}
\maketitle
\section{Internals of \libstfinv\ engines}
While the \texttt{doxygen} generated documentation gives the relation in a
implementation specific quantities, this document uses the end-user
perspective.

\subsection{Fourier domain least squares}
Let $\Fd_{lk}$ be the complex Fourier expansion coefficient at frequency
$\So_l=l\,\Delta\So$ for the time series recorded at offset $\Sr_k$ to the
source. 
The corresponding Fourier coefficient for the synthetic waveform is
$\Fs\Silk$.
The synthetic waveform is numerically simulated for
some model of subsurface structure and a time history of the source
used to excite the synthetic wavefield which is given by
$N$ Fourier coefficients $\Fg_l$.
$\Fd\Silk$, $\Fs\Silk$, and $\Fg_l$ are complex numbers.

This engine returns filter coefficients 
\begin{equation}
\Fq_l=\frac{\Ssk\Sf^2_k\,\Fs\Silk\Scc\,\Fd\Silk}{M\,\SmE\,\Se^2+
\Ssk\Sf^2_k\,\left|\Fs\Silk\right|^2}
\label{eq:least:squares:solution}
\end{equation}
such that $\FQl=\Fq_l\,\Fg_l$
are Fourier coefficients of an optimized time history of the source.
Here $\Fs\Silk\Scc$ is the complex conjugate of $\Fs\Silk$
and
\begin{equation}
\SmE=\frac{1}{M\,N}\SslN\Ssk\Sf^2_k\,\left|\Fs\Silk\right|^2
\end{equation}
is the average power of the Fourier coefficients $\Fs\Silk$ scaled by
$\Sf_k$.
The optimized time history of the source provides an improved fit of the
synthetics $\Fq_l\,\Fs\Silk$ to the recorded data $\Fd\Silk$ in a
least-squares sense.
This is equivalent to minimizing the least-squares objective function
\begin{equation}
F(\Fq_l;\Se)=\SslN\Ssk\Sf^2_k\,\left|\Fd\Silk-\Fq_l\Fs\Silk\right|^2
+M\,\SmE\,\Se^2\SslN\left|\Fq_l\right|^2
\label{eq:least:squares:error}
\end{equation}
with respect to the real and imaginary parts of all $\Fq_l$.
While eq.~\eqref{eq:least:squares:error} makes the least-squares approach
obvious, eq.~\eqref{eq:least:squares:solution} demonstrates that the solution
is a water-level deconvolution, essentially.

By means of the
scaling coefficients $\Sf_k$ we can make sure that all receivers
$k$ contribute to an equal average amount to
eq.~\eqref{eq:least:squares:error}.
The coefficients could be chosen
\begin{equation}
\Sf_k=\sqrt{\frac{N}{\SslN\left|\Fs\Silk\right|^2}}
\end{equation}
such that the average power
\begin{equation}
\frac{1}{N}\SslN\Sf^2_k\,\left|\Fs\Silk\right|^2=1
\end{equation}
for each $k$.
In the actual implementation we prefer
\begin{equation}
\Sf_k=\left(\frac{\Sr_k}{1\,\text{m}}\right)^\kappa
\end{equation}
using $\kappa$ to adjust a compensation for a power law attenuation with
offset $\Sr$.
Alternatively, $\Sf_k$ can be made large for small $\Sr_k$, such that the
source correction filter primarily is obtained from near-offset records.
This is achieved by choosing $\kappa$ small or even negative and
might by preferable in early stages of the inversion when the model of
the subsurface structure is hardly able to produce waves at correct
propagation velocity.

With $\Se=0$ the $\Fq_l$ in eq.~\eqref{eq:least:squares:error} optimize the
data fit only.
At frequencies where $\Fs\Silk$ become small with respect to the noise in 
$\Fd\Silk$, the solution in eq.~\eqref{eq:least:squares:solution} develops a
singularity (for $\Fs\Silk\rightarrow0$).
A finite $\Se$ is used to keep the least-squares solution regular
by introducing a stabilizing penalty to the objective function in
eq.~\eqref{eq:least:squares:error} and
a water-level to the denominator of eq.~\eqref{eq:least:squares:solution}.
The $\Fq_l$ are damped (kept artificially small) as the
average power 
\begin{equation}
\frac{1}{M}\Ssk\Sf_k^2\left|\Fs\Silk\right|^2
\end{equation}
of the scaled synthetics $\Sf_k\Fs\Silk$ at frequency $\So_l$ becomes smaller
than a fraction $\Se^2$ of the overall average power $\SmE$ of the scaled
synthetics $\Sf_k\Fs\Silk$.
$\Se^2$ is the waterlevel-parameter to be passed to the engine.
\end{document}

% ----- END OF libstfinv.tex ----- 
